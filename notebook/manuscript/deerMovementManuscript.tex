
\documentclass[10pt,a4paper]{article}
\usepackage{f1000_styles}

%% Default: numerical citations
% \usepackage[numbers]{natbib}

%% Uncomment this lines for superscript citations instead
% \usepackage[super]{natbib}

%% Uncomment these lines for author-year citations instead
% \usepackage[round]{natbib}
% \let\cite\citep

%% lines required to use a CSL style for references
% definitions for citeproc citations
\NewDocumentCommand\citeproctext{}{}
\NewDocumentCommand\citeproc{mm}{%
  \begingroup\def\citeproctext{#2}\cite{#1}\endgroup}
\makeatletter
 % allow citations to break across lines
 \let\@cite@ofmt\@firstofone
 % avoid brackets around text for \cite:
 \def\@biblabel#1{}
 \def\@cite#1#2{{#1\if@tempswa , #2\fi}}
\makeatother
\newlength{\cslhangindent}
\setlength{\cslhangindent}{1.5em}
\newlength{\csllabelwidth}
\setlength{\csllabelwidth}{3em}
\newenvironment{CSLReferences}[2] % #1 hanging-indent, #2 entry-spacing
 {\begin{list}{}{%
  \setlength{\itemindent}{0pt}
  \setlength{\leftmargin}{0pt}
  \setlength{\parsep}{0pt}
  % turn on hanging indent if param 1 is 1
  \ifodd #1
   \setlength{\leftmargin}{\cslhangindent}
   \setlength{\itemindent}{-1\cslhangindent}
  \fi
  % set entry spacing
  \setlength{\itemsep}{#2\baselineskip}}}
 {\end{list}}
\usepackage{calc}
\newcommand{\CSLBlock}[1]{\hfill\break#1\hfill\break}
\newcommand{\CSLLeftMargin}[1]{\parbox[t]{\csllabelwidth}{\strut#1\strut}}
\newcommand{\CSLRightInline}[1]{\parbox[t]{\linewidth - \csllabelwidth}{\strut#1\strut}}
\newcommand{\CSLIndent}[1]{\hspace{\cslhangindent}#1}

%% lines to get the code chunks working

%% lines to enable bulletpoints in a new notation style
\providecommand{\tightlist}{%
  \setlength{\itemsep}{0pt}\setlength{\parskip}{0pt}}

\begin{document}
\pagestyle{fancy}

\title{Roe Deer Movement}
\author[1]{Benjamin Michael Marshall*}
\author[2]{---**}
\author[3]{}
\author[4]{}
\author[3]{}
\author[5]{}
\author[6]{}
\author[7]{}
\author[1]{}
\affil[1]{---}
\affil[2]{---}
\affil[3]{}
\affil[4]{}
\affil[5]{}
\affil[6]{}
\affil[7]{}
\affil[*]{\href{mailto:benjaminmichaelmarshall@gmail.com}{\nolinkurl{benjaminmichaelmarshall@gmail.com}}}
\affil[**]{---}

\maketitle
\thispagestyle{fancy}

\begin{abstract}

abstract text

\end{abstract}

\section*{Keywords}

Movement ecology, step selection function, poisson, habitat preference, habitat selection, animal movement, roe deer

\clearpage
\pagestyle{fancy}

\section{Introduction}\label{introduction}

\section{Methods}\label{methods}

\subsection{Study Area}\label{study-area}

\subsection{Home Range Esimation}\label{home-range-esimation}

We estimated roe deer home range using autocorrelated kernel density estimators (\citeproc{ref-ctmm}{Christen H. Fleming and Calabrese 2023}, \citeproc{ref-Fleming2017}{2017}; \citeproc{ref-Calabrese2016}{Calabrese, Fleming, and Gurarie 2016}; \citeproc{ref-Fleming2015}{Christen H. Fleming et al. 2015}).
This process consisted of fitting a number of continuous time movement models to an individuals movement data, selecting the best fitting movement model, and extracting a suitable range contour from the resulting utilisation distribution.
We fit the following models (following the default process provided by the ctmm package): OU, OUF, IID etc here.
We used AIC to determine the best fitting movement model on an individual basis, and used that single best fitting model for all further estimations.

Before committing to the estimations of home range size we examined whether the roe deer exhibited stable ranges through the visual inspection of variograms.
A stable range should be reveal by a clear asymptote in the variogram.
We paired these visual inspections with a judgement of effective sample size to help gasuge our confident in the area estimates (DEFINTION OF ESS HERE WOULD BE GOOD) .
All our individuals showed effective sample sizes \_\_\_\_\_, leading us to be confident in overall home range estimates.

{[}PMREL goes in here somewhere{]} (\citeproc{ref-silva_autocorrelationinformed_2022}{Silva et al. 2022})

Having determine the data suitability for home range estimations, we extracted the 95\% and 99\% contours from the weighted AKDE estimate, alongside 95\% CI surrounding that contour.
We selected 95\% as a balance between a generous estimate of home range, while also avoiding the undue influence of the most extremely outlying movements.
To generate an overall home range estimate for UK roe deer, we averaged all home ranges using the weighted mean function provided by the ctmm package (\citeproc{ref-ctmm}{Christen H. Fleming and Calabrese 2023}; \citeproc{ref-silva_autocorrelationinformed_2022}{Silva et al. 2022}).
This way the home range mean is weighted by the confidence (i.e., ESS) surrounding each home range.

We retained 99\% estimates for guiding the between patch conenctivity to maximise the repeated between patch modelling, ie connecting as many patches as would be likely for an individual roe deer in the course of their life.
We calculated the widest dimension of each home range polygon (or largest polygon if the home range area was non-contiguous), and halved that to serve as a proxy for the distance deer could travel between patches.
A mean of this longest dimension was used to limit which patches could be considered likely to be travelled between by deer.
To confirm this approach, we determined the distance from patch for every deer location that fell outside a patch.
We examined the distribution of these distance values that revealed that 95\% of all deer movements fell within the mean of half longest dimension of the 99\% home range area.

\subsection{Habitat Selection}\label{habitat-selection}

We determined selection using two methods: an integrated step selection function and a poisson model (Muff et al paper CITE).

Both methods require a comparison of use (i.e., GPS locations of the deer) to available points (i.e., randomly generated locations the deer could that travelled).
For each confirmed deer location we generated \_\_ random alternative locations they could have travelled to.
The location of these random locations was governed by two distributions.
A gamma distribution that random step lengths were drawn from, and a von Mises distribution that random turn directions were drawn from.
Both distribution where calibrated (e.g., shape, size, mu, and kappa) by the underlying movement data.
The same data was used in both modelling approaches.

Once all random locations had been generated we extracted a suite of environmental conditions at all those locations.
First was the land use type.
Using the 25m resolution.
{[}NEED SOMETHING ON RESOLUTION{]}
We used the land use data provided by UKCEH (CITE AND DETAILS HERE).
We recategoriesd the UKCEH land use categories into \_\_ more general categories that reduced instances of limited interaction with the deer movement data thereby aiding habitat selection model convergence and avoided extreme, unstable selection estimates.
We also acquired woody linear feature (i.e., hedgerows) data from UKCEH (CITE AND DETAILS).
We converted the polygon spatial data into a raster, where 1 == hedgerow, and used that rasterisation to generate a distance to hedgerow raster for the entire study landscape.
We conducted the same process to create a distance to woodland raster, where we calculated the distance from any area the UKCEH land use data classed as deciduous or coniferous woodland.
These distance rasters allowed for easy extraction of the distance to the nearest hedgerow and woodland for all locations.
We acquired road data from OS map open GOV licensed (CITE HERE with DETAILS).
We created a binary variable describing crossing events for all steps, with all steps that crossed one or more of the roads being classed as 1.
This binary variable allowed us to estimate the likelihood deer cross a road and therefore the level of barrier roads present.

For the iSSFs we ran a single model for each individual deer.
The model formulae consisted of land use (\_\_ category variable, with decidious woodland placed as the reference category), distance to woodland (continous in m), distance to hedgerow (continuous in m), road crossing (binary).
In addition to these selection focused predictors, we several movement predictors including step length, turn angle, log step length, cos turn angle, as well as the interaction between step length and land use, step length and distance to woodland.
The Poisson model was a single population level model that included all the individuals data.
The formula was largely the same as the iSSF, except for the removal of the movement predictors and interactions, and the addition of fixed gaussian process to allow for selection effect to vary by individual.
Previous work has demonstrate that the inclusions of those aid bias reduction in iSSFs but can lead to increased bias and more variable estimates when placed in the Poisson model approach.

\subsection{Translating Preferences into Connectivity}\label{translating-preferences-into-connectivity}

Once we had estimates of selection and effect for all the environmental aspects of interest, we spatially mapped those estimated covariates back onto the landscape, resulting in a map of resistance.
For the iSSF the covariates we used consisted of the median estimate for each predictor, and as some required a required a step length and turn angle, we elected to use the mean of both to generate all resistance maps.
For both methods we ignored the uncertainty surrounding the selection estimates, instead relying on just the point estimates for the resistance mapping.

We used two methods to simulate potential connectivity across the landscapes.
The first was random shortest paths, that consists of generating random walks between locations.
This method was used to show the connectivity of areas used by RIENDEER(?), and offers a mechanism for calibrating the connectivity maps to both the habitat selection and movement path characteristics.
We used habitat patches as the sources of our random locations.
Each patch had \_\_ start and end locations generated within it, we then ran walks from these locations to all other patch locations within 750 m of that patch.
We elected not to generate start or end points in patches less than 10ha in areas, following .
This was due to these patches being unlikely to host deer populations and the reduction in origin patches aided computational costs.
We selected 750 m as that represented the mean longest axis of the 99\% HR estimate of the deer (excluding outlying non-contiguous portions of the area polygons).
Once every walk was complete, the resulting rasters describing the likelihood of a deer crossing a cell are complied into a single landscape raster describing connectivity and standardised between 0 and 1 (where 0 is low connectivity and 1 is high connectivity).
A key consideration in these walks is how random the paths are.
We elected to run walks at \_\_ different levels of randomness (theta; with 1 being close to a least cost path, and \_\_\_\_ being the most random and diffuse walks).
To determine what level of randomness best reflected the realised movements of the deer, we compared the resulting connectivity maps to dynamic Brownian Bridge Movement Models.
Dynamic Brownian Bridge Movement Models run a series of random walks between defined start and end points, from the summation of these walks you can extract a rasterised occurrence distribution.
Critically the dBBMM walks are calibrated to the movement capacity of the animal through rolling window (i.e., a number of data points) that summarises the movement rate during that time.
Additionally within that window, a margin (a subset of data points) is used to detect any sudden changes in movement capacity that may be reflect of behavioural/movement mode changes.
Therefore the dBBMMs provide a estimate of how diffuse the movements could be between known locations.
We ran dBBMMs for all roe deer with a window size of \_\_ and a margin of \_\_, that provided estimates of motion variance on roughly a weekly sliding window with the allowance of sudden motion variance changes day to day (margin).
We compared the dBBMMs to the connectivity maps constructed with varying levels of randomness, and used mean squared least {[}EXTACT TERM HERE{]} to determine which level of randomness best fit the movement data.

\subsection{Validation}\label{validation}

We examined whether the connectivity maps generated matched the observed movements of roe deer using a logistic regression.
The model was supplied with the known locations of deer as well as \_\_\_ randomly generated points across the landscape, all of which had associated connectivity values.
We formulated the model to predict whether a point was used or random based on the connectivity values, and we included a random effect for deer ID.
The expectation was that the model coefficients would indicate that deer locations were positively associated with higher connectivity values.

\section{Results}\label{results}

\section{Discussion}\label{discussion}

\subsection{Conlcusions}\label{conlcusions}

\section{Acknowledgements}\label{acknowledgements}

\section{Software availablity}\label{software-availablity}

For all analysis we used R (v.4.2.2) (\citeproc{ref-base}{R Core Team 2022}), and R Studio (v.2024.09.1+394) (\citeproc{ref-rstudio}{Posit team 2024}). For analysis of animal movement data we used amt (v.0.2.2.0) (\citeproc{ref-amt}{Signer, Fieberg, and Avgar 2019}), ctmm (v.1.2.0) (\citeproc{ref-ctmm}{Christen H. Fleming and Calabrese 2023}), and move (v.4.2.4) (\citeproc{ref-move}{Kranstauber, Smolla, and Scharf 2023}). For general data manipulation we used glue (v.1.7.0) (\citeproc{ref-glue}{Hester and Bryan 2024}), sjmisc (v.2.8.10) (\citeproc{ref-sjmisc}{Lüdecke 2018}), tidyverse (v.2.0.0) (\citeproc{ref-tidyverse}{Wickham et al. 2019}), and units (v.0.8.5) (\citeproc{ref-units}{E. Pebesma, Mailund, and Hiebert 2016}). For project and code management we used here (v.1.0.1) (\citeproc{ref-here}{Müller 2020}), tarchetypes (v.0.9.0) (\citeproc{ref-tarchetypes}{Landau 2021a}), and targets (v.1.9.1) (\citeproc{ref-targets}{Landau 2021b}). For visualisation we used the following as expansions from the tidyverse suite of packages: ggdist (v.3.3.2) (\citeproc{ref-ggdist2024a}{Kay 2024b}, \citeproc{ref-ggdist2024b}{2024a}), ggridges (v.0.5.6) (\citeproc{ref-ggridges}{Wilke 2024}), ggtext (v.0.1.2) (\citeproc{ref-ggtext}{Wilke and Wiernik 2022}), patchwork (v.1.2.0) (\citeproc{ref-patchwork}{Pedersen 2024}), and scales (v.1.3.0) (\citeproc{ref-scales}{Wickham, Pedersen, and Seidel 2023}). Other pacakges we used were boot (v.1.3.28) (\citeproc{ref-boot2021}{Canty and Ripley 2021}; \citeproc{ref-boot1997}{Davison and Hinkley 1997}), crew (v.0.10.2) (\citeproc{ref-crew}{Landau 2024}), and usethis (v.2.2.3) (\citeproc{ref-usethis}{Wickham et al. 2024}). To generate typeset outputs we used bookdown (v.0.41) (\citeproc{ref-bookdown2024}{Xie 2024}, \citeproc{ref-bookdown2016}{2016}), and rmarkdown (v.2.28) (\citeproc{ref-rmarkdown2024}{Allaire et al. 2024}; \citeproc{ref-rmarkdown2018}{Xie, Allaire, and Grolemund 2018}; \citeproc{ref-rmarkdown2020}{Xie, Dervieux, and Riederer 2020}). To manipulate and manage spatial data we used gdistance (v.1.6.4) (\citeproc{ref-gdistance}{van Etten 2017}), raster (v.3.6.26) (\citeproc{ref-raster}{Hijmans 2023}), sf (v.1.0.16) (\citeproc{ref-sf2023}{E. Pebesma and Bivand 2023}; \citeproc{ref-sf2018}{E. Pebesma 2018}), sp (v.2.1.4) (\citeproc{ref-sp2005}{E. J. Pebesma and Bivand 2005}; \citeproc{ref-sp2013}{Bivand, Pebesma, and Gomez-Rubio 2013}), terra (v.1.7.78) (\citeproc{ref-terra}{Hijmans 2024}), and tidyterra (v.0.6.0) (\citeproc{ref-R-tidyterra}{Hernangómez 2023}). To run models and explore model outputs we used effects (v.4.2.2) (\citeproc{ref-effects2019}{Fox and Weisberg 2019}, \citeproc{ref-effects2018}{2018}; \citeproc{ref-effects2003}{Fox 2003}; \citeproc{ref-effects2009}{Fox and Hong 2009}), INLA (v.23.4.24) (@ \citeproc{ref-INLA2013b}{Martins et al. 2013}; @ \citeproc{ref-INLA2015d}{Lindgren and Rue 2015}; \citeproc{ref-INLA2017e}{Rue et al. 2017}; \citeproc{ref-INLA2018f}{Bakka et al. 2018}; \citeproc{ref-INLA2016g}{De Coninck et al. 2016}; \citeproc{ref-INLA2017h}{Verbosio et al. 2017}; \citeproc{ref-INLA2018i}{Kourounis, Fuchs, and Schenk 2018}), lme4 (v.1.1.36) (\citeproc{ref-lme4}{Bates et al. 2015}), and performance (v.0.13.0) (\citeproc{ref-performance}{Lüdecke et al. 2021}).

\section{Data availabilty}\label{data-availabilty}

\section{Author Contributions}\label{author-contributions}

\section{Supplementary Material}\label{supplementary-material}

\clearpage

\section*{References}\label{references}
\addcontentsline{toc}{section}{References}

\phantomsection\label{refs}
\begin{CSLReferences}{1}{0}
\bibitem[\citeproctext]{ref-rmarkdown2024}
Allaire, JJ, Yihui Xie, Christophe Dervieux, Jonathan McPherson, Javier Luraschi, Kevin Ushey, Aron Atkins, et al. 2024. \emph{{rmarkdown}: Dynamic Documents for r}. \url{https://github.com/rstudio/rmarkdown}.

\bibitem[\citeproctext]{ref-INLA2018f}
Bakka, Haakon, Håvard Rue, Geir-Arne Fuglstad, Andrea I. Riebler, David Bolin, Janine Illian, Elias Krainski, Daniel P. Simpson, and Finn K. Lindgren. 2018. {``Spatial Modelling with {INLA}: {A} Review.''} \emph{WIRES (Invited Extended Review)} xx (Feb): xx--. \url{http://arxiv.org/abs/1802.06350}.

\bibitem[\citeproctext]{ref-lme4}
Bates, Douglas, Martin Mächler, Ben Bolker, and Steve Walker. 2015. {``Fitting Linear Mixed-Effects Models Using {lme4}.''} \emph{Journal of Statistical Software} 67 (1): 1--48. \url{https://doi.org/10.18637/jss.v067.i01}.

\bibitem[\citeproctext]{ref-sp2013}
Bivand, Roger S., Edzer Pebesma, and Virgilio Gomez-Rubio. 2013. \emph{Applied Spatial Data Analysis with {R}, Second Edition}. Springer, NY. \url{https://asdar-book.org/}.

\bibitem[\citeproctext]{ref-Calabrese2016}
Calabrese, Justin M., Chris H. Fleming, and Eliezer Gurarie. 2016. {``Ctmm: An {R} {Package} for {Analyzing} {Animal} {Relocation} {Data} {As} a {Continuous}-{Time} {Stochastic} {Process}.''} \emph{Methods in Ecology and Evolution} 7 (9): 1124--32. \url{https://doi.org/10.1111/2041-210X.12559}.

\bibitem[\citeproctext]{ref-boot2021}
Canty, Angelo, and B. D. Ripley. 2021. \emph{{boot}: Bootstrap r (s-Plus) Functions}.

\bibitem[\citeproctext]{ref-boot1997}
Davison, A. C., and D. V. Hinkley. 1997. \emph{Bootstrap Methods and Their Applications}. Cambridge: Cambridge University Press. \url{http://statwww.epfl.ch/davison/BMA/}.

\bibitem[\citeproctext]{ref-INLA2016g}
De Coninck, Arne, Bernard De Baets, Drosos Kourounis, Fabio Verbosio, Olaf Schenk, Steven Maenhout, and Jan Fostier. 2016. {``{Needles}: Toward Large-Scale Genomic Prediction with Marker-by-Environment Interaction.''} \emph{Genetics} 203 (1): 543--55. \url{https://doi.org/10.1534/genetics.115.179887}.

\bibitem[\citeproctext]{ref-Fleming2017}
Fleming, Christen H., and Justin M. Calabrese. 2017. {``A New Kernel Density Estimator for Accurate Home-Range and Species-Range Area Estimation.''} \emph{Methods in Ecology and Evolution} 8 (5): 571--79. \url{https://doi.org/10.1111/2041-210X.12673}.

\bibitem[\citeproctext]{ref-ctmm}
---------. 2023. \emph{{ctmm}: Continuous-Time Movement Modeling}. \url{https://CRAN.R-project.org/package=ctmm}.

\bibitem[\citeproctext]{ref-Fleming2015}
Fleming, Christen H, William F Fagan, Thomas Mueller, Kirk A Olson, Peter Leimgruber, and Justin M Calabrese. 2015. {``Rigorous Home Range Estimation with Movement Data: {A} New Autocorrelated Kernel Density Estimator.''} \emph{Ecology} 96 (5): 1182--88.

\bibitem[\citeproctext]{ref-effects2003}
Fox, John. 2003. {``Effect Displays in {R} for Generalised Linear Models.''} \emph{Journal of Statistical Software} 8 (15): 1--27. \url{https://doi.org/10.18637/jss.v008.i15}.

\bibitem[\citeproctext]{ref-effects2009}
Fox, John, and Jangman Hong. 2009. {``Effect Displays in {R} for Multinomial and Proportional-Odds Logit Models: Extensions to the {effects} Package.''} \emph{Journal of Statistical Software} 32 (1): 1--24. \url{https://doi.org/10.18637/jss.v032.i01}.

\bibitem[\citeproctext]{ref-effects2018}
Fox, John, and Sanford Weisberg. 2018. {``Visualizing Fit and Lack of Fit in Complex Regression Models with Predictor Effect Plots and Partial Residuals.''} \emph{Journal of Statistical Software} 87 (9): 1--27. \url{https://doi.org/10.18637/jss.v087.i09}.

\bibitem[\citeproctext]{ref-effects2019}
---------. 2019. \emph{An r Companion to Applied Regression}. 3rd ed. Thousand Oaks CA: Sage. \url{https://socialsciences.mcmaster.ca/jfox/Books/Companion/index.html}.

\bibitem[\citeproctext]{ref-R-tidyterra}
Hernangómez, Diego. 2023. {``Using the {tidyverse} with {terra} Objects: The {tidyterra} Package.''} \emph{Journal of Open Source Software} 8 (91): 5751. \url{https://doi.org/10.21105/joss.05751}.

\bibitem[\citeproctext]{ref-glue}
Hester, Jim, and Jennifer Bryan. 2024. \emph{{glue}: Interpreted String Literals}. \url{https://CRAN.R-project.org/package=glue}.

\bibitem[\citeproctext]{ref-raster}
Hijmans, Robert J. 2023. \emph{{raster}: Geographic Data Analysis and Modeling}. \url{https://CRAN.R-project.org/package=raster}.

\bibitem[\citeproctext]{ref-terra}
---------. 2024. \emph{{terra}: Spatial Data Analysis}. \url{https://CRAN.R-project.org/package=terra}.

\bibitem[\citeproctext]{ref-ggdist2024b}
Kay, Matthew. 2024a. \emph{{ggdist}: Visualizations of Distributions and Uncertainty}. \url{https://doi.org/10.5281/zenodo.3879620}.

\bibitem[\citeproctext]{ref-ggdist2024a}
---------. 2024b. {``{ggdist}: Visualizations of Distributions and Uncertainty in the Grammar of Graphics.''} \emph{IEEE Transactions on Visualization and Computer Graphics} 30 (1): 414--24. \url{https://doi.org/10.1109/TVCG.2023.3327195}.

\bibitem[\citeproctext]{ref-INLA2018i}
Kourounis, D., A. Fuchs, and O. Schenk. 2018. {``Towards the Next Generation of Multiperiod Optimal Power Flow Solvers.''} \emph{IEEE Transactions on Power Systems} PP (99): 1--10. \url{https://doi.org/10.1109/TPWRS.2017.2789187}.

\bibitem[\citeproctext]{ref-move}
Kranstauber, Bart, Marco Smolla, and Anne K Scharf. 2023. \emph{{move}: Visualizing and Analyzing Animal Track Data}. \url{https://CRAN.R-project.org/package=move}.

\bibitem[\citeproctext]{ref-tarchetypes}
Landau, William Michael. 2021a. \emph{{tarchetypes}: Archetypes for Targets}.

\bibitem[\citeproctext]{ref-targets}
---------. 2021b. {``The Targets r Package: A Dynamic Make-Like Function-Oriented Pipeline Toolkit for Reproducibility and High-Performance Computing.''} \emph{Journal of Open Source Software} 6 (57): 2959. \url{https://doi.org/10.21105/joss.02959}.

\bibitem[\citeproctext]{ref-crew}
---------. 2024. \emph{{crew}: A Distributed Worker Launcher Framework}. \url{https://CRAN.R-project.org/package=crew}.

\bibitem[\citeproctext]{ref-INLA2015d}
Lindgren, Finn, and Håvard Rue. 2015. {``Bayesian Spatial Modelling with {R}-{INLA}.''} \emph{Journal of Statistical Software} 63 (19): 1--25. \url{http://www.jstatsoft.org/v63/i19/}.

\bibitem[\citeproctext]{ref-sjmisc}
Lüdecke, Daniel. 2018. {``{sjmisc}: Data and Variable Transformation Functions.''} \emph{Journal of Open Source Software} 3 (26): 754. \url{https://doi.org/10.21105/joss.00754}.

\bibitem[\citeproctext]{ref-performance}
Lüdecke, Daniel, Mattan S. Ben-Shachar, Indrajeet Patil, Philip Waggoner, and Dominique Makowski. 2021. {``{performance}: An {R} Package for Assessment, Comparison and Testing of Statistical Models.''} \emph{Journal of Open Source Software} 6 (60): 3139. \url{https://doi.org/10.21105/joss.03139}.

\bibitem[\citeproctext]{ref-INLA2013b}
Martins, Thiago G., Daniel Simpson, Finn Lindgren, and Håvard Rue. 2013. {``Bayesian Computing with {INLA}: {N}ew Features.''} \emph{Computational Statistics and Data Analysis} 67: 68--83.

\bibitem[\citeproctext]{ref-here}
Müller, Kirill. 2020. \emph{{here}: A Simpler Way to Find Your Files}. \url{https://CRAN.R-project.org/package=here}.

\bibitem[\citeproctext]{ref-sf2018}
Pebesma, Edzer. 2018. {``{Simple Features for R: Standardized Support for Spatial Vector Data}.''} \emph{{The R Journal}} 10 (1): 439--46. \url{https://doi.org/10.32614/RJ-2018-009}.

\bibitem[\citeproctext]{ref-sp2005}
Pebesma, Edzer J., and Roger Bivand. 2005. {``Classes and Methods for Spatial Data in {R}.''} \emph{R News} 5 (2): 9--13. \url{https://CRAN.R-project.org/doc/Rnews/}.

\bibitem[\citeproctext]{ref-sf2023}
Pebesma, Edzer, and Roger Bivand. 2023. \emph{{Spatial Data Science: With applications in R}}. {Chapman and Hall/CRC}. \url{https://doi.org/10.1201/9780429459016}.

\bibitem[\citeproctext]{ref-units}
Pebesma, Edzer, Thomas Mailund, and James Hiebert. 2016. {``Measurement Units in {R}.''} \emph{R Journal} 8 (2): 486--94. \url{https://doi.org/10.32614/RJ-2016-061}.

\bibitem[\citeproctext]{ref-patchwork}
Pedersen, Thomas Lin. 2024. \emph{{patchwork}: The Composer of Plots}. \url{https://CRAN.R-project.org/package=patchwork}.

\bibitem[\citeproctext]{ref-rstudio}
Posit team. 2024. \emph{{RStudio}: Integrated Development Environment for r}. Boston, MA: Posit Software, PBC. \url{http://www.posit.co/}.

\bibitem[\citeproctext]{ref-base}
R Core Team. 2022. \emph{{R}: A Language and Environment for Statistical Computing}. Vienna, Austria: R Foundation for Statistical Computing. \url{https://www.R-project.org/}.

\bibitem[\citeproctext]{ref-INLA2017e}
Rue, Håvard, Andrea I. Riebler, Sigrunn H. Sørbye, Janine B. Illian, Daniel P. Simpson, and Finn K. Lindgren. 2017. {``Bayesian Computing with {INLA}: {A} Review.''} \emph{Annual Reviews of Statistics and Its Applications} 4 (March): 395--421. \url{http://arxiv.org/abs/1604.00860}.

\bibitem[\citeproctext]{ref-amt}
Signer, Johannes, John Fieberg, and Tal Avgar. 2019. {``Animal Movement Tools (Amt): R Package for Managing Tracking Data and Conducting Habitat Selection Analyses.''} \emph{Ecology and Evolution} 9: 880--90.

\bibitem[\citeproctext]{ref-silva_autocorrelationinformed_2022}
Silva, Inês, Christen H. Fleming, Michael J. Noonan, Jesse Alston, Cody Folta, William F. Fagan, and Justin M. Calabrese. 2022. {``Autocorrelation‐informed Home Range Estimation: {A} Review and Practical Guide.''} \emph{Methods in Ecology and Evolution} 13 (3): 534--44. \url{https://doi.org/10.1111/2041-210X.13786}.

\bibitem[\citeproctext]{ref-gdistance}
van Etten, Jacob. 2017. {``R Package Gdistance: Distances and Routes on Geographical Grids.''} \emph{Journal of Statistical Software} 76 (13): 1--21. \url{https://doi.org/10.18637/jss.v076.i13}.

\bibitem[\citeproctext]{ref-INLA2017h}
Verbosio, Fabio, Arne De Coninck, Drosos Kourounis, and Olaf Schenk. 2017. {``Enhancing the Scalability of Selected Inversion Factorization Algorithms in Genomic Prediction.''} \emph{Journal of Computational Science} 22 (Supplement C): 99--108. \url{https://doi.org/10.1016/j.jocs.2017.08.013}.

\bibitem[\citeproctext]{ref-tidyverse}
Wickham, Hadley, Mara Averick, Jennifer Bryan, Winston Chang, Lucy D'Agostino McGowan, Romain François, Garrett Grolemund, et al. 2019. {``Welcome to the {tidyverse}.''} \emph{Journal of Open Source Software} 4 (43): 1686. \url{https://doi.org/10.21105/joss.01686}.

\bibitem[\citeproctext]{ref-usethis}
Wickham, Hadley, Jennifer Bryan, Malcolm Barrett, and Andy Teucher. 2024. \emph{{usethis}: Automate Package and Project Setup}. \url{https://CRAN.R-project.org/package=usethis}.

\bibitem[\citeproctext]{ref-scales}
Wickham, Hadley, Thomas Lin Pedersen, and Dana Seidel. 2023. \emph{{scales}: Scale Functions for Visualization}. \url{https://CRAN.R-project.org/package=scales}.

\bibitem[\citeproctext]{ref-ggridges}
Wilke, Claus O. 2024. \emph{{ggridges}: Ridgeline Plots in {``{ggplot2}''}}. \url{https://CRAN.R-project.org/package=ggridges}.

\bibitem[\citeproctext]{ref-ggtext}
Wilke, Claus O., and Brenton M. Wiernik. 2022. \emph{{ggtext}: Improved Text Rendering Support for {``{ggplot2}''}}. \url{https://CRAN.R-project.org/package=ggtext}.

\bibitem[\citeproctext]{ref-bookdown2016}
Xie, Yihui. 2016. \emph{{bookdown}: Authoring Books and Technical Documents with {R} Markdown}. Boca Raton, Florida: Chapman; Hall/CRC. \url{https://bookdown.org/yihui/bookdown}.

\bibitem[\citeproctext]{ref-bookdown2024}
---------. 2024. \emph{{bookdown}: Authoring Books and Technical Documents with r Markdown}. \url{https://github.com/rstudio/bookdown}.

\bibitem[\citeproctext]{ref-rmarkdown2018}
Xie, Yihui, J. J. Allaire, and Garrett Grolemund. 2018. \emph{R Markdown: The Definitive Guide}. Boca Raton, Florida: Chapman; Hall/CRC. \url{https://bookdown.org/yihui/rmarkdown}.

\bibitem[\citeproctext]{ref-rmarkdown2020}
Xie, Yihui, Christophe Dervieux, and Emily Riederer. 2020. \emph{R Markdown Cookbook}. Boca Raton, Florida: Chapman; Hall/CRC. \url{https://bookdown.org/yihui/rmarkdown-cookbook}.

\end{CSLReferences}

\end{document}
